\documentclass[a4paper, DIV20, 11pt, headsepline, parskip]{article}
\usepackage[utf8]{inputenc}
\usepackage{enumerate}
\usepackage{amsmath}
\usepackage{amssymb}
\usepackage{mathtools}
\usepackage{fancyhdr}
\usepackage{natbib}
\usepackage{graphicx}
\usepackage{listings} 
\usepackage{hyperref}
\usepackage{pdfpages}
\usepackage{listings}
\usepackage[ngerman]{babel}
\usepackage{xstring}
\usepackage{etoolbox}
\usepackage{listingsutf8}
\lstset{inputencoding=utf8/latin1}
\pagestyle{fancy}
%%%%%%%%%BEGIN CONFIG%%%%%%%%%
\newcommand{\usubject}{Dokumentation (2long.net)}
\newcommand{\uauthor}{Thies Müller,\\ Marcel Tönnies,\\ Robin Schlaak\\}
\newcommand{\udate}{\today}
\newcommand{\uexcnumber}{7}
\newcommand{\uheader}{Müller, Tönnies, Schlaak}
\newcommand{\ugroupe}{}
\newcommand{\urelease}{}
%%%%%%%%%END CONFIG%%%%%%%%%
%%%%%%%%%INTERNAL SETTINGS%%%%%%%%%

%%%%%%%%%%%%%%%%%%%%%%%%%%%%%%%%%%%


%PAGE FOOTER
\lfoot{Autor: Thies Müller - Version: 1.2 -  }


%PAGE HEADER
\lhead{\uheader}
\chead{}
\rhead{\usubject {} \udate}
%%%%%%%%%%%%%%%%%%%%%%%%%%%%%%%%%%%

%%%%%%%%%FRONT PAGE%%%%%%%%%
\title{Dokumentation des Projektes 2long.net}
\author{\uauthor}
\begin{document}
\date{\udate}
\maketitle
\thispagestyle{empty}
\newpage
%%%%%%%%%BEGIN CONTENT%%%%%%%%%
\setcounter{page}{1}
\tableofcontents
\pagebreak
\section{Projekt}

\subsection{Vorstellung des Projekts}
Wir haben uns entschlossen ein Portal zu erstellen in dem (ehemalige) Schüler der Berufsbildenden Schule 2 aus Wolfsburg Bücher verleihen oder verkaufen können.
Als Vertrauensstufe und Verwaltende Personen sollen die Lehrer/innen der BBS2 fungieren. 
\subsection{IST-Analyse}
Zu dem Momentanen Zeitpunkt gibt es die Möglichkeit Einträge in das interne Forum auf dem iServ zu erstellen.
Dies ist aber nur begrenzt sinnvoll einsetzbar, da kurz nach dem Ausscheiden aus der Schule der iServ Account gelöscht wird.
So gibt es für ehemalige Schüler nicht die Möglichkeit ihre alten Schulbücher zu verkaufen.
\subsection{Soll-Konzept}
Wir wollen eine vom iServ unabhängige Plattform entwickeln wo sich Schüler der BBS2 Wolfsburg registrieren können um ihre Schulbücher weiter zu geben oder diese Kurzzeitig zu verleihen
\subsection{Zeitplanung}
\begin{tabular}{cc}
 Beschreibung & eingeplante Zeit\\
 \hline
 Prüfung des Lastenheftes & 5 Stunden\\
 Erstellen eines Pflichtenheftes & 3 Stunden\\
 Einrichten / Konfigurieren eines Webservers mit SSL Verschlüsselung & 1 Stunde\\
 Aufsetzen und einrichten einer Wordpress Umgebung & 3 Stunden\\
 Anpassen der Wordpress Installation & 30 Stunden\\
 Erstellen einer Dokumentation & 4 Stunden\\
\end{tabular}
\subsection{Kostenkalkulation}
Dadurch das der Webserver bereits lief und die implementierten Plugins alle kostenfrei Verfügbar sind, entstanden keine Projektspeziefischen Kosten.
\section{Realisierung}
\subsection{Vorbereitungen}
Für die Realisierung haben wir einige Möglichkeiten in Betracht gezogen und uns dann für eine Wordpress Installation in Kombination mit Plugins entschieden, welche den Verkauf vereinfachen sollen.
\subsection{Technische Implementierung}
\subsubsection{Technischer Stand}
Technisch ist das Projekt auf einen Webserver mit nginx 1.6.2,
MySQL 5.6.23-1 basierend auf einem
Debian 3.16.7-ckt25-2 (2016-04-08)x86-64 GNU/Linux 
realisiert.
Wir haben uns entschlossen, da wir mit Nutzerdaten arbeiten eine SSL-Verschlüsselung einzusetzen um uns und die Besucher vor Man-in-the-Middle Attacken oder ähnlichem zu schützen.
Die Konfiguration des nginx Webservers sowie die Konfiguration der SSL Verschlüsselung befindet sich im Anhang.
\subsubsection{Installation und Konfiguration}
Wir haben uns entschieden ein zu dem Zeitpunkt aktuelles Wordpress (4.5.2) in Deutscher Sprache zu installieren.
Bei der Installation wurde der Account "administrator"\ mit der E-Mail Adresse "noreply@2long.net"\ angelegt.
Dieser wird nur für die Einrichtung benutzt.
Die Administration kann dann über personenbezogene Administratoraccounts durchgeführt werden.
\subsubsection{Anpassung von Wordpress mit Plugins und eigenen Modifikationen}
Nach der Installation von Wordpress wurde das Plugin "GarageSale"\ von "Leo Eibler"\ installiert.
Dieses bietet Nutzern die Möglichkeit nach der Registrierung an der Webseite selber Gegenstände zum Verkauf anzubieten.
Das Plugin wurde an unsere Bedürfnisse angepasst und auf die Startseite eingebunden, sodass alle Artikel die Verkauft werden direkt eingesehen werden können.
Bei den Änderungen des Plugins "GarageSale"\ wurde darauf geachtet nicht die Lizenz (Apache 2.0) zu verletzen.

Als weiteres Plugin wurde "Contact Form Add"\ von "umarbajwa"\ installiert.
Dieses bietet die Möglichkeit möglichst simpel ein Kontaktformular in eine Wordpress Installation einzufügen.

Als letztes Plugin kommt "User Role Editor"\ von "Vladimir Garagulya"\ zum Einsatz.
Dieses Plugin nutzen wir dafür um eigene Nutzerrollen in Wordpress zu definieren.
So wurde Beispielsweise die Rolle "Abonnent"\ in "Schüler"\ umbenannt und es wurde eine weitere Rolle mit dem Namen "Lehrer"\ definiert die dazu berechtigt ist "Abonnenten"\ oder "Schüler"\ anzulegen.
Die Benutzerrolle "Administrator"\ wurde nicht editiert um ein späteren Berechtigungsfehler durch ein fehlen der Benutzerrolle auszuschließen.

Als Design wurde die Wordpress Komponente "Tiny Framework"\ von "Thomas Mackevicius"\ installiert.
Dieses ist sowohl für Mobilgeräte als auch Desktopgeräte optimiert und überzeugte durch seinen einfach gestalteten Aufbau auf einem HTML5-Grundgerüst.
Des weiteren bietet das Tiny Framework die Möglichkeit auf Basis des Frameworks selber so genannte "Child-Designs"\ zu definieren, was eine grafische Umgestaltung der Webseite offen lässt.
\section{Testphase}
Das Projekt wurde sowohl einem White-Box Test als auch einem Black-Box Test unterzogen.
Dafür wurde einmal von dem Entwickler des Projektes weitere Benutzeraccounts angelegt ("thies.mueller"\ und "test.lehrer"\ ) und mit diesen die einzelnen Funktionsabläufe überprüft.
Als Black-Box Test wurde ein Zugang an den Projektpartner Robin Schlaak ausgegeben der selbsttätig die Software erforscht hat.
\subsection{Funktionstests}
Folgende Funktionen wurden Positiv getestet:
\begin{itemize}
\item Ein Angebot einstellen
\item Ein Angebot ändern
\item Ein Angebot löschen
\item Ein Angebot auf den Status "Verkauft"\ setzen
\item Ein Angebot löschen
\item Ein Angebot als Administrator löschen
\item Ein Angebot als Administrator editieren
\item Ein Bild zu einem Angebot hinzufügen
\item Eine Kontaktanfrage stellen
\item Eine Registrierung ausführen
\item Einen Nutzeraccount bestätigen
\end{itemize}
\subsection{Aufgetretene Probleme}
Während der Testphase trat das Problem auf das eine Benachrichtigung per E-Mail an die Lehrkräfte bei einer neuen Nutzer Registrierung nicht möglich ist.
Die E-Mail kann nur an einen Administrator verschickt werden.
Mittelfristig wird geprüft ob eine Weiterleitung von der an den Administrator Account geschickten Mail, über den Mail Server Sinn ergibt.
Diese Entscheidung übergeben wir mit der Dokumentation an den Kunden.
\section{Zusammenfassung}
Zusammenfassend wurde das Ziel eine Plattform zu entwickeln auf der Schüler und ehemalige Schüler der Berufsbildenden Schulen 2 aus Wolfsburg Bücher und andere Schulmaterialien untereinander weiter Verkaufen können, falls sie zum Beispiel aus der Schule ausscheiden.
\section{Ausblick}
Langfristig könnte überlegt werden eine Zahlungsabwicklung über die Plattform zu realisieren.
Dies war keine Anforderung des Lasten- und/oder Pflichtenheftes und wurde deshalb nicht implementiert.
Eine weitere Idee wäre ein Bewertungssystem einzuführen, anhand dessen man Nutzer mit denen man gehandelt hat Bewerten kann.
Auch eine Möglichkeit direkt über die Plattform den Verkäufer zu kontaktieren und die Abwicklung des Verkaufes darüber zu gestalten ist denkbar.
Dafür fehlen aber einige Funktionen im für das Projekt genutzte Grundgerüst (Wordpress)
\section{Anhang}
In dem Anhang sind Konfigurations Dateien die für die Installation unter gleichen Umständen benötigt werden.
\subsection{nginx Konfiguration}
\lstinputlisting[breaklines=true]{nginx.conf}
\subsection{ssl Konfiguration}
\lstinputlisting[breaklines=true]{ssl.conf}
\subsection{Wordpress Konfiguration}
Aus dieser Konfiguration wurden die Zugangsdaten zu dem MySQL Server entfernt.
\lstinputlisting[language=php,breaklines=true]{input.php}
\pagebreak
\section{Quellen}
\begin{itemize}
\item Wordpress
\begin{itemize}
\item \url{https://de.wordpress.org/latest-de_DE.zip}
\end{itemize}
\item Contact Form
\begin{itemize}
\item Autor: umarbajwa (WEB-SETTLER)
\item \url{https://wordpress.org/plugins/contact-form-add/}
\end{itemize}
\item GarageSale
\begin{itemize}
\item Autor: Leo Eibler
\item \url{https://wordpress.org/plugins/garagesale/}
\end{itemize}
\item User Role Editor
\begin{itemize}
\item Autor: Vladimir Garagulya
\item \url{https://wordpress.org/plugins/user-role-editor/}
\end{itemize}
\item Tiny Framework
\begin{itemize}
\item Autor: Tomas Mackevicius
\item \url{https://wordpress.org/themes/tiny-framework/}
\end{itemize}
\end{itemize}

%%%%%%%%%END CONTENT%%%%%%%%%%%
\end{document}